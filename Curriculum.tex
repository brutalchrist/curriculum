%% Inicio del archivo `template-es.tex'.
%% Copyright 2006-2013 Xavier Danaux (xdanaux@gmail.com).
%
% This work may be distributed and/or modified under the
% conditions of the LaTeX Project Public License version 1.3c,
% available at http://www.latex-project.org/lppl/.

\documentclass[11pt,a4paper,roman]{moderncv}   % opciones posibles incluyen tamaño de fuente ('10pt', '11pt' and '12pt'), tamaño de papel ('a4paper', 'letterpaper', 'a5paper', 'legalpaper', 'executivepaper' y 'landscape') y familia de fuentes ('sans' y 'roman')
\usepackage[utf8]{inputenc}
% temas de moderncv
\moderncvstyle{casual}                        % las opciones de estilo son 'casual' (por omision),'classic', 'oldstyle' y 'banking'
\moderncvcolor{grey}                          % opciones de color 'blue' (por omision), 'orange', 'green', 'red', 'purple', 'grey' y 'black'

%\nopagenumbers{}                             % elimine el 

% ajustes para los margenes de pagina
\usepackage[scale=0.75]{geometry}
%\setlength{\hintscolumnwidth}{3cm}           % si desea cambiar el ancho de la columna para las fechas

% datos personales
\name{Sebastián Ernesto}{González Villena}
%%\title{Título del CV (opcional)}
\address{Cuevas 1995 ~ Santiago}{Santiago}
\phone[mobile]{+56~(9)~97358426}                     
%\phone[fixed]{+56~(2)~678~901}                      
\email{sgonzalez@debianchile.cl}                                 
\homepage{https://github.com/brutalchrist}
%%\extrainfo{informacion adicional}
\photo[64pt][0.0pt]{picture}
%%\quote{Alguna cita (opcional)}

% para mostrar etiquetas numericas en la bibliografia (por omision no se muestran etiquetas), solo es util si desea incluir citas en en CV
%\makeatletter
%\renewcommand*{\bibliographyitemlabel}{\@biblabel{\arabic{enumiv}}}
%\makeatother

% bibliografia con varias fuentes
%\usepackage{multibib}
%\newcites{book,misc}{{Libros},{Otros}}
%----------------------------------------------------------------------------------
%            contenido
%----------------------------------------------------------------------------------
\begin{document}
\maketitle
\section{Antecedentes personales}
\cventry{RUT}{16.727.053-0}{}{}{}{}
\cventry{Nacimiento}{18 de Abril de 1988}{}{}{}{}

\section{Formación académica}
%\cventry{2007--2009}{Ingenier�a Civil Inform�tica}{Universidad Cat�lica del Maule}{Talca}{}{Incompleta}
\cventry{2010--2013}{Ingeniería en Informática}{Universidad Autónoma de Chile}{Talca}{}{}
\cvitem{Proyecto Título}{\emph{Creación e implementación de Línea Base para una empresa}}
\cvitem{Descripción}{Definición de parámetros de configuración para las estaciones de trabajo de una empresa, generando una distribución personalizada de GNU/Linux}

\section{Formación complementaria}
\cventry{2015}{Certificación académica}{Scrum Manager Nivel Experto}{Santiago}{}{}

\section{Habilidades}
	\cvitemwithcomment{Backend}{}{PHP, NodeJS, Java}
	\cvitemwithcomment{Frontend}{}{Angular 4, TypeScript, Javascript, jQuery, Ajax}
	\cvitemwithcomment{DB}{}{Oracle, MySQL, Postgres}


\section{Experiencia Laboral}
\cventry{2017--Actual}{Ingeniero}{Contraloría General de la República}{Santiago}{}{
\begin{itemize}%
	\item GEO Portal Municipal: Sistema de disponibilización de datos municipales georeferenciados.
	\begin{itemize}%
		\item Backend: Java (API Rest).
		\item Frontend:  Angular 4 (seed).
		\item Base de datos Oracle.
	\end{itemize}
	\item SIANTE: Sistema de análisis territorial.
	\begin{itemize}%
		\item Backend: Java (API Rest).
		\item Frontend:  Javascript.
		\item Base de datos Oracle.
	\end{itemize}
	\item SISGEOB: Sistema de Gestión de Contratos de Obras.
	\begin{itemize}%
		\item Backend: Java (API Rest).
		\item Frontend:  Angular 4, HTML, jQuery, Javascript y Ajax.
		\item Base de datos Oracle.
		\item Servido en GNU/Linux corriendo en WebLogic.
	\end{itemize}
\end{itemize}
}

\cventry{2015--2017}{Desarrollador de Software}{LemonTech}{Santiago}{}{Empresa dedicada al desarrollo de soluciones para abogados.\newline{}%
\begin{itemize}%
	\item The Timebilling: Software de gestión legal.
	\begin{itemize}%
		\item Backend: PHP.
		\item Frontend:  HTML, jQuery, Javascript y Ajax.
		\item Base de datos MySQL.
		\item Servido en GNU/Linux corriendo en Apache2.
		\item Control de versiones GIT, utilizando plataforma GitHub.
	\end{itemize}
	\item Competencias generales.
	\begin{itemize}%
		\item Desarrollo ágil utilizando SCRUM.
		\item Entorno cultural tecnológico fomentando el aprendizaje comunitario a través de charlas semanales, pair programming y desarrollos extra-laborales.
	\end{itemize}
\end{itemize}
}

\cventry{2014--2015}{Desarrollador de Software}{Guadaltel Chile}{Santiago}{}{Empresa dedicada al desarrollo de aplicaciones gubernamentales con Software Libre.\newline{}%
\begin{itemize}%
	\item Siapertra: Sistema de tramitación automática de actos administrativos.
	\begin{itemize}%
		\item Backend: Java. 
		\item Frontend:  JSF y Richfaces.
		\item Base de datos Oracle utilizando ORM Hibernate.
		\item Servido en GNU/Linux corriendo sobre WebLogic.
	\end{itemize}
	\item Sagir2 Biobio: Sistema de gestión de inversión regional.
	\begin{itemize}%
		\item Backend: Java.
		\item Frontend:  JSF y Primefaces.
		\item Base de datos Postgres utilizando ORM Hibernate.
		\item Servido en GNU/Linux corriendo sobre JBoss.
	\end{itemize}
	\item Servicios Web IPS.
	\begin{itemize}%
		\item Servido en GNU/Linux corriendo sobre JBoss Fuse.
		\item Control de versiones SVN.
	\end{itemize}
	\item Competencias generales.
	\begin{itemize}%
		\item Desarrollo ágil utilizando SCRUM.
		\item Intercomunicación de la empresa (España) y los clientes (Chile).
	\end{itemize}
\end{itemize}
}

\cventry{2010}{Desarrollador de Software}{VisionLabs Ltda.}{Talca}{}{Empresa dedicada al desarrollo de aplicaciones con Visión Computacional.\newline{}%
\begin{itemize}%
\item Limpet: Software de vigilancia inteligente.
  \begin{itemize}%
  \item Desarrollo de escritorio utilizando C++ e interfaz gráfica Qt.
  \item Utilización de librería de visión computacional OpenCV.
  \item Base de datos MySQL.
  \end{itemize}
\end{itemize}
}

\cventry{2013}{Alumno practicante}{Constructora Galilea S.A.}{Talca}{}{Practica orientada a la administración de servidores GNU/Linux (Squid, Openfire, Oracle Linux) y desarrollo de módulos en Python. \newline{}
\begin{itemize}%
\item Sistemas internos.
  \begin{itemize}%
  \item Sistemas desarrollados en Python con interfaz gráfica GTK.
  \item Base de datos Postgres utilizando ORM interno.
  \end{itemize}
\item Daemon: Desarrollo de Daemon para intercomunicación de sistemas internos y ERP JD Edwars.
  \begin{itemize}%
  \item Desarrollado en C.
  \item Corriendo bajo GNU/Linux (Oracle RedHat).
  \end{itemize}
\item Gestión de servidores GNU/Linux.
  \begin{itemize}%
  \item Distribución CentOS.
  \item Instalación y configuración de servidor de mensajería Openfire.
  \item Instalación y configuración de proxy Squid.
  \end{itemize}
\end{itemize}
}

\cventry{2012--2013}{Relator y fundador Taller de Arduino}{Universidad Autónoma de Chile}{Talca}{}{ 
\begin{itemize}%
\item Temáticas. 
  \begin{itemize}%
  \item Principios básicos de electrónica.
  \item Estructura y uso de una placa Arduino.
  \item Sensores (inputs) y elementos de despliegue (outputs).
  \item Programación de un Arduino.
  \end{itemize}
\item Proyecto final de taller.
  \begin{itemize}%
  \item Robot automóvil:  Automóvil capaz de avanzar por un terreno liso evitando los obstáculos presentados.
  \end{itemize}
\end{itemize}
}




%\section{Idiomas}
%\cvitemwithcomment{Ingles}{Básico}{Lectura lenguaje técnico Informático, comunicación básica}

%\section{Competencias profesionales}
%\cvitemwithcomment{Desarrollo escritorio}{Intermedio}{Qt, C++, C, GTK, Python, Bash, Java}
%\cvitemwithcomment{Desarrollo web}{Intermedio}{HTML, PHP, JQuery, JSF}
%\cvitemwithcomment{Bases de datos}{Intermedio}{MySQL, PosgrestSQL}
%\cvitemwithcomment{Sistemas operativos}{Avanzado}{GNU/Linux}
%\cvitemwithcomment{Electrónica}{Intermedio}{Arduino, Circuitos integrados}

%\section{Intereses}
%\cvitem{Electrónica}{Desarrollo de placas electrónicas mediantes circuitos integrados. Proyectos utilizando placa de desarrollo Arduino.}
%\cvitem{Open source}{Participar en comunidades Open Source. Dictar charlas en eventos relacionados.}
%\cvitem{}{}

\renewcommand{\listitemsymbol}{-~} 

\end{document}

